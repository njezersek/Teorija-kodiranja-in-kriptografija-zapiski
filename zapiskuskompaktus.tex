\input{config.tex}
\begin{document} 

\begin{multicols}{5}
\setlength{\premulticols}{1pt}
\setlength{\postmulticols}{1pt}
\setlength{\multicolsep}{1pt}
\setlength{\columnsep}{2pt}

\section*{Kriptosistem}
\begin{align*}
	\mathcal{B} &\dots \text{besedila} \\
	\mathcal{C} &\dots \text{kriptogrami} \\
	\mathcal{K} &\dots \text{ključi} \\
	\mathcal{E} = \{E_k : \mathcal{B} \to \mathcal{C}; k \in \mathcal{K} \} &\dots \text{kodirne f.} \\
	\mathcal{D} = \{D_k : \mathcal{C} \to \mathcal{B}; k \in \mathcal{K} \} &\dots \text{dekodirne f.} \\
\end{align*}
Za vsak $e \in \mathcal{K}$ obstaja $d \in \mathcal{K}$
\[ D_d(E_e(x)) = x \quad \forall x \in \mathcal{B}\]

Vsaka kodrirna funkcija $E_k \in \mathcal{E}$ je injektivna.

\subsection*{Produkt kriptosistemov}
Naj bosta $\mathcal{S}_1 = (\mathcal{B}_1, \mathcal{C}_1, \mathcal{K}_1, \mathcal{E}', \mathcal{D}')$ in 
$\mathcal{S}_2 = (\mathcal{B}_2, \mathcal{C}_2, \mathcal{K}_2, \mathcal{E}'', \mathcal{D}'')$ kriptosistema za katera
je $\mathcal{C}_1 = \mathcal{B}_2$.

\[ S_1 \times S_2 = (\mathcal{B}_1, \mathcal{C}_2, \mathcal{K}_1 \times \mathcal{K}_2, \mathcal{E}, \mathcal{D})\]
\begin{align*}
	E_{(k_1, k_2)} (x) &= E''_{k_2}(K'_{k_1}(x)) \\
	D_{(k_1, k_2)} (y) &= D'_{k_1}(D''_{k_2}(y))
\end{align*}

\subsection*{Prevedljivost kriptosistemov}
Kripto sistem $\mathcal{S} = (\mathcal{B}, \mathcal{C}, \mathcal{K}, \mathcal{E}, \mathcal{D})$ je prevedljiv na
$\mathcal{S}' = (\mathcal{B}, \mathcal{C}, \mathcal{K}', \mathcal{E}', \mathcal{D}')$, če obstaja $f: \mathcal{K} \to \mathcal{K}'$, 
da za vsak $k \in \mathcal{K}$ velja:
\[ E_k = E'_{f(k)} \qquad D_k = D'_{f(k)} \]
Tedaj pišemo $S \to S'$.

Kriptosistema sta \textbf{ekvivalentna}, če velja $S \to S'$ in $S' \to S$.

Tedaj pišemo $S \equiv S'$.

\subsection*{Idempotentnost kriptosistemov}
Kriptosistem $S$ je idempotenten, če
\[ S \times S \equiv S\]
\textit{Klasični kriposistem so vsi idempotentni.}

\section*{Klasični kriptosistem}
\subsection*{Cezarjeva šifra}
\[ \mathcal{B} = \mathcal{C} = \mathcal{K} = \mathbb{Z}_{25}\]
\[ E_k(x) \equiv x + k \mod 25\]
\[ D_k(y) \equiv y - k \mod 25\]

\subsection*{Substitucijska šifra}
\[ \mathcal{B} = \mathcal{C} = \mathbb{Z}_{25}, \quad \mathcal{K} = S(\mathbb{Z}_{25})\]
Ključ je permutacija $\pi \in \mathcal{K}$
\[ E_k(x) = \pi(x) \]
\[ D_k(y) = \pi^{-1}(y) \]

\subsection*{Afina šifra}
\[ \mathcal{B} = \mathcal{C} = \mathbb{Z}_{25}, \quad \mathcal{K} = \mathbb{Z}_{25}^{*} \times \mathbb{Z}_{25} \]
Ključ $(a, b) \in \mathcal{K}$
\[ K_{(a,b)}(x) = ax + b \mod 25\]
\[ D_{(a,b)}(y) = a^{-1}(y - b) \mod 25\]

\subsection*{Vigenerjeva šifra}
\[ \mathcal{B} = \mathcal{C} = \mathcal{K} = \mathbb{Z}_{25}^n\]
Ključ $\underline{k} \in \mathcal{K}$
\[ K_{\underline{k}}(\underline{x}) = \underline{x} + \underline{k} \mod 25\]
\[ D_{\underline{k}}(\underline{y}) = \underline{y} - \underline{k} \mod 25\]

\subsection*{Permutacijska šifra}
\textit{Simbolov ne nadomeščamo, ampak jih premešamo}
\[ \mathcal{B} = \mathcal{C} = \mathbb{Z}_{25}^n, \quad \mathcal{K} = S_n\]
\[ K_{\pi}(\underline{x}) = \underline{x}_{\pi(1)} + \dots + \underline{x}_{\pi(n)}  \]
\[ D_{\pi}(\underline{x}) = \underline{x}_{\pi^{-1}(1)} + \dots + \underline{x}_{\pi^{-1}(n)}  \]

\subsection*{Hillova šifra}
\[ \mathcal{B} = \mathcal{C} = \mathbb{Z}_{25}^n, \quad \mathcal{K} = \{ A \in \mathbb{Z}_{25}^{n\times n} | \det(A) \in \mathbb{Z}_{25}^* \}\]
Ključ je matrika $A \in \mathcal{K}$
\[ K_A(\underline{x}) = A \underline{x} \mod 25\]
\[ D_A(\underline{y}) = A^{-1} \underline{y} \mod 25\]

\section*{Bločne šifre}
Kripotsistem $(\mathcal{B}, \mathcal{C}, \mathcal{K}, \mathcal{E}, \mathcal{D})$ je bločna šifra dolžine n, 
če je $\mathcal{B} = \mathcal{C} = \Sigma^n$, kjer je $\Sigma$ končna abeceda.

Vsaka kodirna funkcija je ekvivalentna neki permutaciji $\Sigma^n$, njena dekodirna funkcija pa inverzu te permutacije.

\subsection*{Afina bločna šifra}
\[ \Sigma = \mathbb{Z}_m \]
\[ \mathcal{K} = \left\{ (A, \underline{b});\ A \in \mathbb{Z}_m^{n\times n}, \det(A) \in \mathbb{Z}_m^*, \underline{b} \in \mathbb{Z}^n_m \right\}  \]
\begin{align*}
	E_{(A, \underline{b})}(\underline{x}) &\equiv A \underline{x} + \underline{b} \mod m \\
	D_{(A, \underline{b})}(\underline{x}) &\equiv A^{-1} \underline{x} - \underline{b} \mod m 
\end{align*}

\subsection*{Iterativne šifre}
Sestavlja jih 

\begin{itemize}
	\item \textbf{razpored ključev}: Naj bo $K$ ključ. $K$ uporabimo za konstrukcijo krožnih ključev $(K^1,\dots, K^{N_r})$ temu seznamu pravimo razpored ključev.
	\item \textbf{krožna funkcija}: ima dva argumenta: tekoče stanje in krožni ključ:
	\[ w^r = g(w^{r-1}, K^{r}) \]
	Da je dešifriranje možno mora biti $g$ injektivna za vsak fiksen ključ $K$; tj. $\exists g^{-1}:$
	\[ g^{-1}(g(w, K), K) = w \qquad \forall w, K\]
	\item \textbf{šifriranje skozi $N_r$ podobnih krogov}: Besedilo $x$ vzamemo za začetno stanje $w^0$:
	\[ y = g(g(\dots g(g(x, K^1), K^2) \dots, K^{N_r - 1}), K^{N_r})\]
	\item \textbf{dešifriranje}:
	\[ x = g^{-1}(\dots g^{-1}(g^{-1}(y, K^{N_r}), K^{N_r - 1}) \dots, K^1)\]
\end{itemize}


\subsection*{Substitucijsko-permutacijsko omrežje (SPN)}
je iterativna bločna šifra kjer je $\Sigma = \{0,1\}$, $\ell, m \in \mathbb{N}$ in $\mathcal{B} = \mathcal{C} = \Sigma^{\ell m}$

\begin{itemize}
	\item \textbf{substitucije}: $\pi_s \in S(\Sigma^\ell)$ \\
	\textit{S-škatla} - zamenja $\ell$ bitov z drugimi biti
	\item \textbf{permutacije}: $\pi_p \in S_{\ell m}$ \\
	\textit{P-škatla} - zamenja $\ell m$ bitov z drugimi biti
\end{itemize}

\textit{Oznaka za delitev na zloge dolžine $\ell$:}
\[ x = x_1 x_2 \dots x_m, \quad |x_i| = \ell \]

\textbf{Kodiranje:}
\begin{koda}
$w^0 = b$
za $r = 1, \dots, N_r - 1$ :
	$u^r = w^{r-1} \oplus K^r$ // primasamo K
	za $i = 1, \dots, m$ :
		$\underline{v}_i^r = \pi_s(\underline{u}_i^r)$ // substitucija zlogov
	$w^r = v_{\pi_p(1)}^r, \dots, v_{\pi_p(\ell m)}^r$ // permutacija bitov
// zadnji krog
$u^{N_r} = w^{N_r - 1} \oplus K^{N_r}$
za $ i = 1, \dots, m$ :
	$\underline{v}_i^{N_r} = \pi_s(\underline{u}_i^{N_r})$
vrni $c = v^{N_r} \oplus K^{N_r + 1}$  // beljenje
\end{koda}

\textbf{Dekodiranje}:
\begin{koda}
$v^N_r = c \oplus K^{N_r + 1}$
za $ i = 1, \dots, m$:
	$\underline{u}_i^{N_r} = \pi_s^{-1}(\underline{v}_i^{N_r})$
za $ r = N_r - 1, \dots, 1$:
	$w^r = u^r \oplus K^{r+1}$
	$v^r = (w_{\pi_p^{-1}(1)}^r, \dots, w_{\pi_p^{-1}(\ell m)}^r)$
	za $ i = 1, \dots, m$:
		$\underline{u}^r_i = \pi_s^{-1}(\underline{v}_i^r)$
$b = u^1 \oplus K^1$
\end{koda}

\subsection*{Feistelova šifra}
je bločna iterativna šifra dolžine $2t$ za abecedo $\Sigma = \{0, 1\}$.

$N_r$ je št. krogov, $K^1, \dots, K^{N_r}$ razpored ključev, ki ga dobimo iz ključa
$K$ in $f_K: \Sigma^t \to \Sigma^t$ je \textit{Feistelova kodirna funkcija}.

\textit{En krog kodiranja:}
\begin{center}
	\resizebox*{0.5\columnwidth}{!}{\includegraphics{img/feistel.png}}
\end{center}

\textit{Kodiranje}
\begin{koda}
$L_0 = $ leva polovica $b$
$R_0 = $ desna polovica $b$
za $i = 1, \dots, N_r$:
	$L_i = R_{i-1}$
	$R_i = L_{i-1} \oplus f_{K_i}(R_{i-1})$
$c = R_{N_r} \| L_{N_r}$
\end{koda}

\subsection*{DES in AES}
TO-DO!

\section*{Tokovne šifre}
Besedilo $b$ razdelimo na bloke $b = b_1 \dots b_t \in \mathcal{B}^t$.

Imamo zaporedje (tok) ključev: $z_1, z_2, \dots \in \mathcal{K}$.

\textit{Kodiranje}
\begin{koda}
za $j = 1, \dots, t$:
	$c_j = E_{z_j}(b_j)$
$c = c_1 c_2 \dots c_t \in \mathcal{C}^t$
\end{koda}

\textit{Dekodiranje}
\begin{koda}
za $j = 1, \dots, t$:
	$b_j = D_{z_j}(c_j)$
$b = b_1 b_2 \dots c_t \in \mathcal{B}^t$
\end{koda}

\subsection*{Aditivne tokovne šifre}
Naj bo $(G, +)$ grupa, $\mathcal{B} = \mathcal{C} = \mathcal{K}$ in $z_1, z_2, \dots$ tok ključev.

\textit{Kodiranje}
\begin{align*}
E_{z_i} (b_i) &= b_i + z_i \\	
D_{z_i} (c_i) &= c_i - z_i
\end{align*}

\subsection*{Samokodirna šifra}
$\mathcal{B} = \mathcal{C} = \mathcal{K} = \mathbb{Z}_{26}$

Začetni ključ izberemo $z_1 \in \mathbb{Z}_{26}$
\[ z_i = b_{i-1} \quad \text{za }\ i > 1 \]

\textit{Kodiranje}
\[ E_{Z_i}(b_i) = b_i + z_i \]

\textit{Dekodiranje}
\[ D_{Z_i}(c_i) = c_i - z_i \]

\subsection*{Vermanova šifra}
$\mathcal{B} = \mathcal{C} = \mathcal{K} = \{0, 1\}^n$, ključ izberemo naključno.

\textit{Kodiranje}
\[ E_k(b) = b \oplus k \]
\textit{Dekodiranje}
\[ D_k(c) = c \oplus k \]

\textit{To je pravzaprav Vigenerjeva šifra, le da ima ključ enako dolžino kot besedilo}

\textit{Uporabimo kratko seme za generiranje dolgega toka psevdonaključnih bitov, ki jih uporabimo za ključ.}

\subsection*{Linearna rekurzivna šifra}
je sinhrona tokovna šifra, pri kateri je
\[ \mathcal{B} = \mathcal{C} = \mathcal{K} = \mathbb{Z}_s\]
zaporedje ključev z linearno rekurzinvo enačbo reda $m$ s konstantnimi koeficienti nad $\mathbb{Z}_s$:
\[ z_i = c_1 z_{i-1} + c_2 z_{i-2} + \dots + c_m z_{i_m} \mod s \]
Zaporedju lahko priredimo polinom:
\[ C(x) = 1 + \sum_{i=1}^m c_i x^i \mod s\]

\textit{Kodiranje/Dekodiranje:}
\begin{align*}
	E_{z_i}(x_i) &= x_i + z_i \mod s \\
	D_{z_i}(y_i) &= y_i - z_i \mod s
\end{align*}

Perioda LFSR reda $m$ je največ $2^m - 1$

Red nerazcepnega polinoma $f(x)$ je najmanjši $t$, da $f(x) | x^t - 1$.

Če ima LFSR nerazcepen karakteristični polinom reda $t$, potem ima LFSR periodo $t$.

\subsection*{Pomični register z linearno povratno zanko}
V pomičnem registru je na začetku inicializacijski vektor $(z_1 z_2 \dots z_m)$ (ključ).

\begin{center}
	\resizebox*{0.9\columnwidth}{!}{\includegraphics{img/lfsr.png}}
\end{center}

Na vsakem koraku izpišemo $z_1$ register pomaknemo v levo zadnji bit $z_m$ pa 
izračunamo kot z $c_1, \dots, c_m$ uteženo vsoto.

Če poznamo $z_0, \dots, z_{2m-1}$, lahko rešimo sistem:
\[
\begin{bmatrix}
	z_0 & z_1 & \dots & z_{m-1} \\
	z_1 & z_2 & \dots & z_{m-2} \\
	\vdots & \vdots & & \vdots \\
	z_{m-1} & z_{m} & \dots & z_{2m-2} 
\end{bmatrix}
\begin{bmatrix}
	c_m \\
	c_{m-1} \\
	\vdots \\
	c_1
\end{bmatrix}
=
\begin{bmatrix}
	z_m \\
	z_{m+1} \\
	\vdots \\
	z_{2m-1}
\end{bmatrix}
\]
Če smo pravilno uganili red $m$ ima sistem enolično rešitev.

\section*{Asimetrična kriptografija}
\subsection*{RSA}
$n = pq$ kjer sta $p$ in $q$ različni veliki praštevili.

$m = \varphi(n) = (p-1)(q-1)$

Potem je kriptosistem podan z:
\begin{align*}
	\mathcal{B} &= \mathcal{C} = \mathbb{Z}_n \\
	\mathcal{K} &= \{n\} \times \mathbb{Z}_m^* \\
	E_{(n,e)}(x) &\equiv x^e \mod n \\
	E_{(n,d)}(y) &\equiv y^d \mod n
\end{align*}

\textit{$e$ mora biti tuj $m$}

Kodirnemu ključu $(n, e)$ pripada dekodirni ključ $(n, d)$, kjer je $d = e^{-1} \in \mathbb{Z}_m^*$

\subsection*{Problem diskretnega logaritma}
Naj bo $G$ multiplikativna grupa. Za dana $\alpha, \beta \in G$, kjer je red elementa $\alpha$ enak $n$, 
je treba poiskati takšen $x \in \{0, \dots, n-1\}$, da je
\[ \alpha^x = \beta\]
Številu $x$ rečemo diskretni logaritem elementa $\beta$ z osnovo $\alpha$.

\subsubsection*{Shanksov algoritem (veliki korak - mali korak) }
\begin{koda}
vhod: $G$ grupa, $\alpha, \beta \in G$, $n = \text{red}(\alpha)$
izhod: $x = \log_\alpha \beta$
$m = \lceil \sqrt{n} \rceil$
za $j = 0, \dots, m-1$:
	$(j, \alpha^{m-j}) \to L_1$
uredi $L_1$ po drugi komponenti
za $i = 0, \dots, m-1$:
	$(i, \beta \alpha^{-i}) \to L_2$
uredi $L_2$ po drugi komponenti
poisci $(j, y) \in L_1$ in $(i, y) \in L_2$
$x = (mj + i)$
vrni $x$
\end{koda}

\subsection*{Diffie-Hellmanova izmenjava ključev}
\begin{itemize}
	\item Alenka in Bojan se dogovorita za veliko praštevilo $p$ in $\alpha \in \mathbb{Z}_p^*$, ki ima velik red $n$.
	\item Alenka si izbere naključno število $a \in \{1, \dots, n-1\}$,
	izračuna $A = \alpha^a \mod p$ in pošlje $A$ Bojanu.
	\item Bojan si izbere naključno število $b \in \{1, \dots, n-1\}$, 
	izračuna $B = \alpha^b \mod p$ in pošlje $B$ Alenki.
	\item Alenka in bojan vsak zase izračunata skupni tajni ključ $K = \alpha^{ab} = A^b = B^a$
\end{itemize}
\textit{Varnost temelji na težavnosti diskretnega logaritma.}

\textit{Zaradi možnosti napada srednjega moža je pri izmenjavi ključev nujna avtentikacija!}

\subsection*{ElGamalov kriptosistem}
\begin{itemize}
	\item Alenka in Bojan izmenjata tajni ključ k z Diffie-Hellmanovo shemo
	\item Alenka želi poslati sporočilo $x$. Izračuna kriptogram $y = k\cdot x \mod p$ in ga pošlje Bojanu.
	\item Bojan izračuna $x = k^{-1} \cdot y \mod p$
\end{itemize}
Formalna definicija:
\begin{align*}
	\mathcal{B} &= \mathcal{C} = \mathbb{Z}_p^* \\
	\mathcal{K} &= \mathbb{Z}_p^* \times \mathbb{Z}_p^* \\
	E_{(a,B)}(x) &\equiv B^a \cdot x \mod p \\
	D_{(b,A)}(y) &\equiv A^{p-b-1} \cdot y \mod p \\
\end{align*}

Naj bo $B = \alpha^b \mod p$ in $A = \alpha^a \mod p$. Potem kodirnemu kjluču $(a, B)$ ustreza dekodirni ključ $(b, A)$.

\section*{Zgoščevalne funkcije}
Zgoščevalna funkcija besedilu poljubne dolžine kratek izvleček.

Želene lastnosti:
\begin{itemize}
	\item \textbf{Naključnost}: Če se dve sporočili razlikujeta na enem samem mestu morata povzetka izgledati kot neodvisno izbrani naključni števili.
	\item \textbf{Odpornost praslik}: za poljuben izvleček $z$ je računsko nemogoče poiskati sporočilo $x$, ja je $h(x) = z$. Oz. zgoščevalna funkcija je \textbf{enosmerna}.
	\item \textbf{Odpornost drugih praslik}: za dano sporočilo $x$ je nemogoče najti drugo sporočilo $x'$, ki ima enak izvleček.
	\item \textbf{Odpornost na trke}: računsko je nemogoče poiskati dve različni sporočili $x$ in $x'$ z enakim povzetkom.
\end{itemize}
\textbf{Trk} je par različnih sporočil z enakim povzetkom.

\subsection*{Tipična zgoščevalna funkcija}
\begin{itemize}
	\item \textbf{Komprsijska funkcija}: $f: \{0,1\}^{r+n} \to \{0,1\}^n$
	\item \textbf{Zgoščevalna funkcija}: $h: \{0,1\}^* \to \{0,1\}^n$
\end{itemize}

Zgoščevalna funkcija iterativno kliče kompresijsko funkcijo.

\begin{koda}
$H_0 = IV$
za $i = 1, \dots, t$:
	$H_i = f(H_{i-1} \| x_i)$
$h(x) = H_t$
\end{koda}

Tukaj je $IV$ začetno stanje, $x_i$ pa so bloki besedila.

Na konec besedila dodano nekaj bitov, ki popisujejo dolžino besedila in toliko ničel, da se besedilo lahko razdeli na enako velike bloke.

\textit{Če je kompresijska funkcija odporna na trke, je tudi zgoščevalna funkcija odporna na trke.}


\section*{Digitalni podpisi}
Formalno je \textbf{sistem na digitalno podpisovanje} peterka $(\mathcal{B}, \mathcal{A}, \mathcal{K}, \mathcal{S}, \mathcal{V})$, kjer je
\begin{itemize}
	\item $\mathcal{B}$ končna množica sporočil
	\item $\mathcal{A}$ končna množica podpisov
	\item $\mathcal{K}$ končna množica ključev
	\item za vsak ključ $K \in \mathcal{K}$ obstaja algoritem za podpisovanje in preverjanje podpisa
	\[ \text{sig}_K \in S, \qquad \text{sig}_K : \mathcal{B} \to \mathcal{A} \]
	\[ \text{ver}_K \in S, \qquad \text{ver}_K : \mathcal{B} \times \mathcal{A} \to \{\text{true}, \text{false}\} \]
\end{itemize}
Algoritem za podpisovanje je znan le podpisniku.

\subsection*{Podpisovanje z algoritmom RSA}
Naj bosta $p, q$ praštevili in $n = pq$. Naj bo $(n, d)$ zasebni in $(n, e)$ javni ključ.
Potem za $K = (n, e, d)$ definiramo:
\begin{align*}
\text{sig}_K(x) &= x^d \mod n \\
\text{ver}_K(x,y) &= \left( \text{true} \iff x = y^e \mod n \right)
\end{align*}

\subsection*{ElGamalov sistem za digitalno podpisovanje}
\subsubsection*{Generiranje ključa}
Naj bo $p$ takšno praštevilo, ja je v $\mathbb{Z}$ težko izračunati diskretni logaritem in $\alpha \in \mathbb{Z}_p^*$ primitivni element.

Potem je $\mathcal{B} = \mathbb{Z}_p^*$, $\mathcal{A} = \mathbb{Z}_p^* \times \mathbb{Z}_{p-1}$ in 
$\mathcal{K} = \{(p, \alpha, a, \beta) : \beta \equiv \alpha^a \mod p \}$.

Število $a$ je zasebno. Števila $p$, $\alpha$ in $\beta$ pa so javna.

\subsubsection*{Podpisovanje}
Podpisnik s ključem $K = (p, \alpha, a, \beta)$ izbere naključno skriteo število $k \in \mathbb{Z}_{p-1}^*$ in določi
\[ \text{sig}_K(x, k) = (\gamma, \delta)\]
kjer je
\begin{align*}
	\gamma &\equiv \alpha^k \mod p \\
	\delta &\equiv (x - a\gamma) k^{-1} \mod p
\end{align*}

\subsubsection*{Preverjanje podpisa}
Za to potrebujemo $p$, $\alpha$ in $\beta$, ki so javni:
\[ \text{ver}_K(x, \gamma, \delta) = \left( \text{true} \iff \beta^\gamma \gamma^\delta \equiv_p \alpha^x \right)\]

\subsection*{Digital Signature Standard (DSA)}
\subsubsection*{Generiranje ključa}
\begin{itemize}
	\item Izberi 160-bitno praštevilo $q$
	\item Izberi 1024-bitno praštevilo $p$, da $q|(p-1)$
	\item Izberi element $h \in \mathbb{Z}_p^*$ in izračunaj $\alpha = h^{(p-1)/q} \mod p$; ponavljaj dokler $\alpha \neq 1$. ($\alpha$ je generator natanko določen ciklične grupe red $q$ v $\mathbb{Z}_p^*$)
	\item Izberi naključno naravno število $a < q$
	\item Izračunaj $\beta = \alpha^a \mod p$
	\item Janvi ključ osebe $A$ je $(p, q, \alpha, \beta)$, zasebni pa $a$.
\end{itemize}
\textit{Opomba:} red $\alpha, \beta, \gamma$ je enak $q$.

\subsubsection*{Podpisovanje}
\begin{itemize}
	\item Izberi naključno naravno število $k$, ki je manjše od $q$.
	\item Izračunaj $\gamma = \left( \alpha^k \mod p\right) \mod q$
	\item Izračunaj $k^{-1} \mod q$.
	\item Izračunaj $\delta = k^{-1}(h(x) + a\gamma) \mod q$, kjer je $h(x)$ povzetek sporočila $x$, dobljen z zgoščevalno funkcijo SHA-1.
	\item Če je $\gamma = 0$ ali $\delta = 0$, začni ponovno.
	\item Podpis sporočila je $(\gamma, \delta)$.
\end{itemize}

\subsubsection*{Preverjanje podpisa}
\begin{itemize}
	\item Priskirbi si overjeno kopijo javnega kjluča $(p,q,\alpha,\beta)$ podpisnika
	\item Izračunaj $w = \delta^{-1} \mod q$ in $h(x)$
	\item Izračunaj $e_1 = h(x)w \mod q$ in $e_2 = \gamma w \mod q$
	\item Izračunaj $v = (\alpha^{e_1} \beta^{e_2} \mod p ) \mod q $
	\item Sprejmi podpis, če je $v = \gamma$
\end{itemize}


\section*{Uporaba bločnih šifer}
\subsection*{Elektronska kodna knjiga (ECB)}
Naivni način uporabe bločnih šifer. Z istim klučem kodiramo zaporedoma bolk po blok.

\begin{align*}
	c_i &= E_k(b_i) \\
	b_i &= D_k(c_i)
\end{align*}

\subsection*{Veriženje kodnih blokov (CBC)}
Izberemo inicializacijski vektor $IV$ dolžine $n$.

\textit{Kodiranje}
\begin{koda}
$c_0 = IV$
za $j = 1, \dots, m$:
	$c_j = E_e(b_j \oplus c_{j-1})$
$c = c_1  \dots c_m$
\end{koda}

\textit{Dekodiraje}
\begin{koda}
$c_0 = IV$
za $j = 1, \dots, m$:
	$b_j = D_e(c_j) \oplus c_{j-1}$
$b = b_1  \dots b_m$
\end{koda}

Napaka na bloku $c_j$ vpliva le na $b_j$ in $b_{j+1}$

\subsection*{Način s števcem (CM)}
Izberemo števec $ctr$ dolžine $n$. Besedilo razdelimo na bloke dolžine $n$: $b = b_1 \dots b_m$.

\textit{Kodiranje}
\begin{koda}
za $ j = 1, \dots, m$:
	$l_j = ctr + j - 1 \mod 2^n$
	$c_j = b_j \oplus E_e(l_j)$
$c = c_1 \dots c_m$
\end{koda}

\textit{Dekodiranje}
\begin{koda}
za $ j = 1, \dots, m$:
	$l_j = ctr + j - 1 \mod 2^n$
	$b_j = c_j \oplus E_e(l_j)$
$b = b_1 \dots b_m$
\end{koda}

\section*{Napadi na kriptosisteme}
\subsection*{Pasivni napadi}
\begin{itemize}
	\item \textbf{Napad za golim kriptogramom}: nasprotnik pozna enega ali več kriptogramov.
	\item \textbf{Napad z znanim besedilom}: nasprotnik pozna enega ali več parov (besedilo, kriptogram).
	\item \textbf{Napad z izbranim besedilom}: nasprotnik ima začasen dostop do kodirnega postopka. Generira pare $(b,c)$ za izbrana besedila $b$. \textit{V primeru kriptosistemov z javnimi ključi tak napad štejemo za paseiven.}
\end{itemize}

\subsection*{Aktivni napadi}
\begin{itemize}
	\item \textbf{Napd z izbranim kriptogramom}: nasprotnik za izbrane kriptograme lahko zahteva ustrezna besedila. Kasneje dobi kriptogram $c$, ki ga želi dekodirat.
	\item \textbf{Prilagodljivi napad z izbranim kriptogramom}: nasprotnik skuša dešifirati $c$ med tem lahko za izbrane kriptograme lahko zahteva ustrezna besedila.
\end{itemize}

\subsection*{Stopnje varnosti}
\begin{itemize}
	\item \textbf{Brezpogojna varnost}: tudi če ima napadalec neomejene računske vire, samo iz kriptograma na izve nobene informacije o besedilu (razen dolžine)
	\item \textbf{Semantična varnost}: napadalec s polinomsko omejenimi viri samo iz kriptograma z nezanemarljivo verjetnostjo ne izve nobene informacije o besedilu (razen dolžine).
	\item \textbf{Polinomska varnost}: napadalec s polinomsko omejenimi viri z nezanemarljivo verjetnostjo ne more ločiti med kriptogramoma danih besedil iste dolžine.
\end{itemize}
\textit{Za pasivnega napadalca sta semantična in polinomska varnost ekvivalentni.}

\subsection*{Sistemi s popolno tajnostjo (LPT)}
Simetrični kriptosistem $\mathcal{S} = (\mathcal{B}, \mathcal{C}, \mathcal{K}, \mathcal{E}, \mathcal{D})$ opremimo
z verjetnostno porazdelitvijo na množici $\mathcal{B} \times \mathcal{K}$

\begin{align*}
	B \quad &\dots \quad \text{slučajna sprem. z zalogo vrednosti } \mathcal{B} \\
	K \quad &\dots \quad \text{slučajna sprem. z zalogo vrednosti } \mathcal{K} \\
	C \quad &\dots \quad \text{slučajna sprem. z zalogo vrednosti } \mathcal{C} \\
\end{align*}
$C$ je določena z $B$ in $K$

Predpostavimo, da st $B$ in $K$ neodvisni:
\[ P(B = b \cap K = k) = P(B = b) P(K = k) \]
za vsak $b \in \mathbb{B}$ in vsak $c \in \mathcal{C}$ velja še:
\[ P(B = b) > 0 \quad \text{oziroma} \quad P(C = c) > 0 \]

Potem ima kriptosistem $S$ \textbf{lastnost popolne tajnosti} natanko tedaj, ko
\begin{align*}
	\forall b \in \mathcal{B}, c \in \mathcal{C}:& \ P(B = b | C = c) = P(B = b) \\
	\text{exvivalentno} & P(C = c | B = b) = P(C = c)
\end{align*}

Vrednost $C$ za dana $b \in \mathcal{B}$ in $k \in \mathcal{K}$ je:
\[ c = E_k(b) \]

Verjetnost dogodka $(C = c)$ dobimo iz formule za popolno verjetnost:
\[ P(C = c) = \sum_{b \in \mathcal{B}} P(C = c | B = b) P(B = b) \]
\[ P(C = c | B = b) = \sum_{k \in \mathcal{K}:\ E_k(b) = c} P(K = k) \]

\subsubsection*{Verjetnostne formule}
\begin{align*}
	P(A|B) &= \frac{P(A \cap B)}{P(B)} & P(A|B) &= \frac{P(B|A)P(A)}{P(B)} 
\end{align*}

\textit{Trditev:} Če ima kriptosistem lastnost popolne tajnosti, za 
vsak $b \in \mathcal{B}$ in $c \in \mathcal{C}$ obstajaj $k \in \mathcal{K}$, 
da velja $E_k(b) = c$. In $|\mathcal{B}| \leq |\mathcal{C}| \leq |\mathcal{K}|$


\textit{Izrek (Shannon):} Naj velja $|\mathcal{B}| = |\mathcal{C}| = |\mathcal{K}|$.
Potem ima kriptosistem $S$ lastnost popolne tajnosti natanko tedaj, ko
\begin{itemize}
	\item za vsak $b \in \mathcal{B}$ in vsak $c \in \mathcal{C}$ obstaja en $k \in \mathcal{K}$,
	da je $E_k(b) = c$
	\item slučajna spremnljivka $K$ je enakomerno porazdeljena.
\end{itemize}

\section*{Teorija števil}

\subsection{Eulerjeva funkcija}
Eulerjeva funkcija nam pove koliko je obrnlivih elementov v $\mathbb{Z}_m$.

\[ | \mathbb{Z}_m^* | = \varphi(m) \]

Za $n \in \mathbb{N}$ s paraštevilskim razcepom \\ $ n = p_1^{\alpha_1} \cdot ... \cdot p_m^{\alpha_m}$ velja:
\[\varphi(n) = \varphi(p_1^{\alpha_1}) \cdot ... \cdot \varphi(p_m^{\alpha_m}) = n \prod_{ p_k \in \mathbb{P}} \left(1-\frac{1}{p_k} \right) \]

\textbf{Euljerjev izrek:}

Naj bo $G$ končna grupa. Potem red elementa $a \in G$ deli red grupe $G$.

\[\textrm{gcd}(a, m) = 1 \Leftrightarrow a^{\varphi(m)} \equiv_m 1; a \in \mathbb{Z}_m^*\]
\[a,m \in \mathbb{N} \wedge \textrm{gcd}(a, m) = 1 \Rightarrow a^{\varphi(m)} \equiv_m 1\]
\[a^{\varphi(m)} = 1 \text{ v } \mathbb{Z}_m^*\]

\textbf{Mali Fermatov izrek:} če je $m \in \mathbb{P}$ ($\varphi(m) = m-1$) in $\textrm{gcd}(a,m) = 1$, potem:
\[a^{m-1} \equiv_m 1\]

\subsubsection{Fermantov test praštevilskosti}
$p$ praštevilo $\implies$ $a^{p-1} \equiv_p 1$

Če želimo preveriti ali je $p$ praštevilo, zgornjo trditev preizkusimo za nekaj naključnih $a$-jev.

\subsubsection*{Miller-Rabinov test}
Če je $n$ praštevilo mora veljati:

Naključno število $a$ je tuje $n$.

Če zapišemo $n - 1 = 2^s d$, kjer je $d$ liho število, velja eno izmed:
\begin{itemize}
	\item $a^d \equiv_n 1$
	\item $\exists r \in {0,1,\dots, s-1}$, da je $a^{2^r d} \equiv_n -1$
\end{itemize}

\textit{Verjetnost (napake), da zgornje velja za sestavljeno število je največ $\frac{1}{4}$.}

\subsection{Linearne diofantske enačbe}
Diofantska enačba $ax + by = c$ ima rešitev $\Leftrightarrow$ $gcd(a, b) | c$. 

Če ima eno rešitev $(x_0, y_0) \in \mathbb{Z}^2$ ima neskončno množico rešitev:
\[\{(x_k, y_k) : k \in \mathbb{Z}\}\]
\[x_k = x_0 - k\frac{b}{\textrm{gcd(a, b)}}\]
\[y_k = y_0 + k\frac{a}{\textrm{gcd(a, b)}}\]

\subsubsection*{Razširjen evklidov algoritem}

\begin{koda}
vhod: $(a, b)$
($r_0$, $x_0$, $y_0$) = ($a$, 1, 0)
($r_1$, $x_1$, $y_1$) = ($b$, 0, 1)
$i$ = 1

dokler $r_i$ $\neq$ 0:
    $i$ = $i$+1
    $k_i$ = $r_{i-2} // r_{i-1}$
    $(r_i, x_i, y_i)$ = $(r_{i-2}, x_{i-2}, y_{i-2}) - k_i(r_{i-1}, x_{i-1}, y_{i-1})$
konec zanke
vrni: $(r_{i-1}, x_{i-1}, y_{i-1})$
\end{koda}

Naj bosta $a, b \in \mathbb{Z}$. Tedaj trojica $(d, x, y)$, ki jo vrne razširjen evklidov algoritem z vhodnim podatkomk $(a, b)$, zadošča:
\[ax + by = d \text{ in } d = \textrm{gcd}(a, b)\] 


\subsection*{Grupe}
\begin{itemize}
    \item \textbf{grupoid} $(M, \cdot)$ urejen par z neprazno množico $M$ in zaprto opreacijo $\cdot$.
    \item \textbf{polgrupa} grupoid z asociativno operacijo $ \forall x,y,z \in M : (x\cdot y)\cdot z = x\cdot (y\cdot z)$.
    \item \textbf{monoid} polgrupa z enoto $ \exists e \in M \ \forall x \in M : e\cdot x = x\cdot e = x$.
    \item \textbf{grupa} polgrupa v kateri ima vsak element inverz $ \forall x \in M \ \exists x^{-1} \in M : x\cdot x^{-1} = x^{-1}\cdot x = e$.
    \item \textbf{abelova grupa} grupa s komutativno operacijo $ \forall x,y \in M  : x\cdot y = y\cdot x$.
\end{itemize} 

\subsubsection{Množica $\mathbb{Z}_m$}
$\mathbb{Z}_m = \{0,1,...,m-1\}$

Vpeljemo seštevanje $+_m$ po modulu $m$ in množenje $\cdot_m$ po modulu $m$. 
Dobimo grupo $(\mathbb{Z}_m, +_m)$ in monoid $(\mathbb{Z}_m, \cdot_m)$.

Red elementa $x\in \mathbb{Z}_m$ je $\frac{m}{\gcd(m,x)}$

\subsubsection{Množica $\mathbb{Z}_m^*$}
To je množica vseh obrnljivih elementov v $\mathbb{Z}_m$ (operacija: množenje).
\[|\mathbb{Z}^*_m| = \varphi(m)\]
Element $x\in \mathbb{Z}_m$ je obrnljiv če se da rešiti \emph{diofantsko enačbo}:
\[ xy + km = 1\]
za neznanki $y$ (inverz od $x$) in $k$.

\subsubsection{Cayleyjeva tabela}
Za vsak element množice imamo en stolpec in eno vrstico. V vsakem polju je produkt elementa vrstice in elementa stolpca.
(Presek vrstice $a$ in stolpca $b$ je $ab$)

\subsubsection{Red elementa}
Naj bo $(G,\cdot)$ grupa. Red elemneta $a$ je najmanjše naravno število $n \in \mathbb{N}$, da velja
\[a^n = e\]
\textit{oznaka:} $\#a$

\subsubsection{Red grupe}
je število elementov $G$, oznaka $|G|$.

\subsubsection*{Ciklična grupa}
Grupa je ciklična, če vsebuje $a$ reda $|G|$:
\[ G = \left\{ a, a^2, a^3, \dots, a^{|G|} = e\right\}\]

\subsubsection*{Grupa $\mathbb{Z}_p^*$}
\[ ( \mathbb{Z}_p^*, \cdot ) \cong (\mathbb{Z}_{p-1}, +)\]
\[ \text{red}_{\mathbb{Z}_p^*}(\alpha^i) = \text{red}_{\mathbb{Z}_{p-1}}(i) = \frac{p-1}{\gcd(i, p-1)}\]

$x$ je generator grupe $\mathbb{Z}^*_p$ $\iff$ $\# x = p - 1$

$x$ je generator grupe $\mathbb{Z}^*_p$ $\iff$ $x^{\frac{p-1}{p_i}} \neq 1 \mod p$, za vsak $i$, jer je $p-1 = p_1^{k_1} \dots p_l^{k_l}$.

\subsection*{Končni obsegi}
$(K, +,\cdot)$ je obseg, če je
\begin{itemize}
	\item $(K, +)$ abelova grupa
	\item $(K^*, \cdot)$ grupa ($K^* = K \setminus \{0\}$)
	\item velja distributivnost:
	\[ a \cdot (b+c) = (a\cdot b) + (a \cdot c)\]
	\[ (a+b) \cdot c = (a\cdot c) + (b \cdot c)\]
\end{itemize}

Obseg je \textbf{komutativen}, če je $(K^*, \cdot)$ komutativna.

\subsection*{Praštevilski obsegi}
Če je $p$ praštevilo, je $(\mathbb{Z}_p, +_p, \cdot_p)$ končen obseg.


\subsection*{Galoisovi obsegi}
\[\text{GF}(p) \cong \mathbb{Z}_p \qquad p \in \mathbb{P}\]
\[ \text{GF}(p^n) \cong \mathbb{Z}_p[x]/(u) \]
\begin{itemize}
	\item $u \in \mathbb{Z}_p[x]$ je nerazcepen polinom stopnje $n$
	\item elementi $\text{GF}(p^n)$ so ostanki polinomov iz $\mathbb{Z}_p$ pri deljenju z polinomom $u$
	\item seštevanje je enako kot seštevanje v $\mathbb{Z}_p[x]$
	\item produkt izračunamo v $\mathbb{Z}_p[x]$ nato pa vzamemo ostanek pri deljenju z $u$
\end{itemize}

Množica neničelnih/obrnljivih elementov $(GF(p^n)^*, \cdot) \cong (\mathbb{Z}_{p^n-1}, \cdot)$ je vedno izomorfna neki ciklični grupi.
Generatorjem te grupe rečemo \textbf{primitivni elementi} Galoisovega obsega.

\subsection*{Kitajski izrek o ostankih}
Naj bodo $n_1, \dots, n_k$ paroma tuja.
\begin{align*}
	x &\equiv a_1 \mod n_1 \\
	 & \vdots \\
	x &\equiv a_k \mod n_k \\
\end{align*}
\[ N = n_1 \cdot n_2 \cdot \dots \cdot n_k \]
Vse rešitve zgornjega sistema so kongurentne po modulu $N$.

\[ N_i = \frac{N}{n_i} \qquad M_i = \text{ inverz } N_i \text{ po modulu } n_i\]
\[x = \sum_{i=1}^{k} a_i M_i N_i \mod N\]



\newpage
\section*{Kodi}
Naj bo $\Sigma$ končna abeceda. Definirajmo $\Sigma^* = \Cup_{n=0}^\infty \Sigma^n$.

\textbf{Kod} $\mathcal{C}$ nad abecedo $\Sigma$ je končna podmnožica $\Sigma^*$ ($\mathcal{C}\subset \Sigma^*$)

\begin{itemize}
	\item \textbf{Kodiranje} je preslikava $f: \mathcal{S}\rightarrow\mathcal{C}$
	\item Po prenosu po komunikacijskem kanalu prejmemo besedo $y$.
	\item Če $y \notin \mathcal{C}$, ji po nekeem pravilu priredimo besedo
		$x\prime \in \mathcal{C}$. Pravimo, da besedo \textbf{dekodiramo}.
\end{itemize}


\section{Bločni kodi}
Kod $\mathcal{C}$ nad abecedo $\Sigma$ je \textbf{bločni kod dolžine $n$}, 
če imajo vse kodne besede dolžino $n$ ($\mathcal{C} \subseteq \Sigma^n$).

\textbf{Hammingova razdalja} med besedama $x$ in $y$ je definirana kot
\[ d(x,y) = |\{i;x_i\neq y_i\}|\]
\textbf{Teža} besede $x$ ($t(x)$) je definirana kot število neničelnih mest v besedi.

\textbf{Razmaknjenost} koda:
\[d = d(\mathcal{C}) = \min\{d(x,y); x,y\in \mathcal{C}, x \ne y\}\]

\subsection*{Pravila za dekodiranje}

\begin{itemize}
	\item \textbf{Pravilo najmanjše napake} \\
		Prejeto besedo $y$ dekodiramo v tisti $x \in \mathcal{C}$, ki maksimizira
		\[P[x\ \text{oddana}|y\ \text{sprejeta}]\]
		\[P[x|y] = \frac{P[y|x]\cdot P[x]}{P[y]} = \frac{P[y|x]\cdot P[x]}{\sum_{c\in\mathcal{C}}P[y|c]\cdot P[c]}\]
	\item \textbf{Pravilo največje verjetnosti} \\
		Prejeto besedo $y$ dekodiramo v tisti $x \in \mathcal{C}$, ki maksimizira
		\[P[y\ \text{sprejeta}|x\ \text{oddana}]\]
		\textit{Če so vse kodne besede enako verjetne, sta PNN in PNV enaki.}
	\item \textbf{Pravilo najbližjega soseda} \\
		Prejeto besedo $y\in \Sigma^n$ dekodiramo v tisto besedo $x\in \mathcal{C}$, pri kateri je $d(x,y)$ najmanjša.

		\textit{Če je $p<1/2$, dajeta PNN in PNS enak rezultat.}
\end{itemize}


Kod $\mathcal{C}$ označimo z $(n,M,d)$-kod, kjer je
\begin{align*}
	n \quad \dots \quad & \text{bločna doložina} \\
	M \quad \dots \quad & \text{št. kodnih besed} \\
	d \quad \dots \quad & \text{razmaknjenost} \\
\end{align*}


\subsection{Napaka}
$y = x+e$, kjer je $e \in \Sigma^n$ \textbf{napaka}.
\begin{itemize}
	\item Kod \textbf{odkrije} $s$ napak, če $x+e\notin \mathcal{C}$ za vse $x\in \mathcal{C}$
		in vse $e$, za katere je $1\le t(e)\le s$
	\item Kod \textbf{popravi} $s$ napak, če
		\[d(x+e,x)<d(x+e,x')\]
		za vse $x,x'\in\mathcal{C}$ in vse $e\in\Sigma^n$, za katere je $t(e)\le s$.
\end{itemize}

Bločni kod z bločno dolžino $n$ in razmaknjenostjo $d$ \textbf{odkrije} $d-1$ napak in 
\textbf{popravi} $\lfloor \frac{d-1}{2} \rfloor$ napak.


\section{Linearni kodi}
Kod $\mathcal{C} \subseteq \Sigma^n$ je \textbf{linearen}, če je vektorski podprostor:
\[\forall c_1,c_2\in\mathcal{C},\ \alpha,\beta \in \Sigma \implies \alpha c_1 + \beta c_2\in \mathcal{C}\]
Dimenzija koda ($k$) je dimenzija vektorskega podporstora. \\

$[n,k,d]$-kod nad $\Sigma = GF(q)$ je linearen $(n,q^k,d)$-kod.
\[d = \min_{x\in \mathcal{C}, x\neq 0} t(x)\]
\[M = q^k\]
\textbf{Generatorska matrika} $G$ koda $\mathcal{C}$ je matrika velikosti $k\times n$.
Njene vrstice so kodne besede, ki sestavljajo bazo kode (vektorskega podporstora).

\subsection*{Nadzorna matrika}
$\mathcal{C}^\perp = \{x\in\Sigma^n; cx^T=0\ \forall c\in\mathcal{C}\}$ je \textbf{dualni kod} koda $\mathcal{C}$.

Generatorsko matriko koda $\mathcal{C}^\perp $ imenujemo \textbf{nadzorna matrika} koda $\mathcal{C}$

\[G\in \Sigma^{k\times n}, H\in \Sigma^{(n-k)\times n}\]
\[\text{rang}(G) = k, \text{rang}(H) = n-k\]
potem velja: $G$ je generatorska in $H$ nadzorna matrika nekega linearnega koda
$\iff G\times H^T=0$.

\subsection{Sindrom}
\[y = x+e\]
$Hy^T$ imenujemo \textbf{sindrom} besede $y$.
\[Hy^T = He^T\]

\subsection{Kodiranje}
\[c = s G\]

\subsection{Dekodiranje}
\begin{itemize}
	\item Izračunaj $\sigma = Hy^T$
	\item V tabeli $T$ poišči $e$: $He^T = \sigma$
		(če ga ni, zahtevaj ponoven prenos besede)
	\item Vrni $x = y-e$
\end{itemize}
\subsection{Razmaknjenost}
Naj bo $\mathcal{C}$ linearen kod nad abecedo $GF(q)$ z ndzorno matriko $H$.
Potem velja:

$d(\mathcal{C})\ge d \iff $ vsaka množica $d-1$ stolpcev matrike $H$ je linearno neodvisna nad $GF(q)$

$d(\mathcal{C}) = \max\{d;\text{vsakih $d-1$ stolpcev $H$ je lin. neodvisnih}\} $

\subsection*{Ekvivalentnost}
Koda sta \textbf{ekvivalentna}, če lahko iz enega dobimo drugega z zaporedjem transormacij \textbf{kodne matrike}
\begin{itemize}
	\item premutacije stolpcev
	\item premutacije simbolov v izbranem stolpcu
	\item permutacije vrstic
\end{itemize}
\[\mathcal{C}_1\sim\mathcal{C}_2\]
Za vsak $[n,k,d]$-kod obstaja ekvivalenten kod z generatorsko matriko v standardni obliki
\[G = [I_k|A] \qquad H = [-A^T | I_{n-k}]\]

\section{Meje za kode}
\[ A_q(n, d) = \max\{M; \exists (n,M,d)\ \text{kod nad}\ GF(q)\}\]
\[ A_q(n,1) = q^n\]
\[ A_2(n,2) = 2^{n-1}\]
\[|K(x,r)|=\sum_{k=0}^r\binom{n}{k}(q-1)^k\]

\subsection{Hamingova zgornja meja}
\[A_q(n,d) \le \frac{q^n}{\sum_{k=0}^{\lfloor\frac{d-1}{2}\rfloor}\binom{n}{k}(q-1)^k}\]
Če kod dosega Hammingovo mejo, je \textbf{popoln}.

\subsection{Gilbert-Varshamova spodnja meja}
\[A_q(n,d) \ge \frac{q^n}{\sum_{k=0}^{d-1}\binom{n}{k}(q-1)^k}\]

\subsection{Hammingov kod reda $r$}
nad $\Sigma = GF(q)$ je $[n,k,d]$-kod dolžine $n=\frac{q^r-1}{q-1}$ in dimenzije $k = n-r$, podan z
nadzorno matriko $H \in \Sigma^{r\times n}$, v kateri sta vsaka dva stolpca linearno neodvisna.

\textit{Hammingovi kodi so popolni.}

\subsection{Singletonova meja}
Naj bo $\mathcal{C}$ $(n,M,d)$-kod nad $GF(q)$. Potem je $M \le q^{n-d+1}$.


Za linearni $[n,k,d]$-kod je $d\le n-k+1$.

Linearni $[n,k,d]$-kod lahko popravi največ $\lfloor\frac{n-k}{2}\rfloor$ napak.
\section{Ciklični kodi}
Besedo $\hat{x} =x_nx_1\cdots x_{n-1}$ imenujemo \textbf{ciklični pomik} besede $x$.

Linearen kod je \textbf{cikličen}, če velja
\[x\in\mathcal{C}\Rightarrow\hat{x}\in\mathcal{C}\]
Besedo $x=x_1\cdots x_n$ identificiramo s polinomom 
\[x(t) = x_1+x_2t+\cdots +x_nt^{n-1}\in GF(q)[t]/(t^n-1) \]
Besedi $\hat{x}$ ustreza polinom $t\cdot x(t)(\mod t^n-1)$.

Naj bo $\mathcal{C}$ cikličen kod in $g(t)$ neničeln polinom najmanjše stopnje v $\mathcal{C}$.
Potem velja:
\begin{itemize}
	\item $\mathcal{C} = \left<g(t)\right> = \{g(t)\cdot a(t) \mod t^n-1; a(t)\in GF(q)[t]\}$ (ideal, ki ga generira $g(t)$)
	\item $g(t)\ |\ (t^n-1)$
	\item $\text{dim} (\mathcal{C}) = k = n -\text{deg}(g)$ \\ in $B = \{g(t), tg(t),\cdots,t^{k-1}g(t)\}$ je baza $\mathcal{C}$.
\end{itemize}
Ciklični kodi dolžine $n$ nad $GF(q)$ ustrezajo deliteljem polinoma $t^n-1 \in GF(q)[t]$.

Če $\mathcal{C}=\left<g(t)\right>$, imenujemo $g$ \textbf{generatorski polinom} koda $\mathcal{C}$.

\[G=\begin{bmatrix}g(t) \\ tg(t) \\ \vdots \\ t^{k-1} g(t)\end{bmatrix}\]
je generatorska matrika za $\mathcal{C}$.

\[t^{n-k+i} = q_i(t)g(t)+r_i(t)\]
Potem velja:
\[t^{n-k+i}-r_i(t) = q_i(t)g(t)\in \mathcal{C}\]
\[G' =\begin{bmatrix}-r_0(t) & 1 & 0 & \cdots &0\\ -r_1(t) & 0&1&\cdots &0\\ \vdots \\
-r_{k-1}(t)&0&0&\cdots&1\end{bmatrix}\] je potem generatorska matrika za $\mathcal{C}$.
\subsection{Kodiranje}
\[t^{n-k}s(t) = q(t)g(t)+r(t)\implies x(t) = t^{n-k}\cdot s(t)-r(t)\in \mathcal{C}\]

Sporočilo $s(x)$ pomnožimo z generatorskim polinomom $g(x)$
\[ x(t) = s(t) g(t)\]


\section{Reed-Solomonovi kodi}
Naj bodo $\alpha_1,\alpha_2,\ldots,\alpha_p\in GF(q)$ in $\alpha_i\neq 0, \alpha_i\neq\alpha_j$
\[ g(t)=(t-\alpha_1)(t-\alpha_2)\cdots(t-\alpha_p)\ |\ (t^{q-1}-1)\]
\textit{Posledica:} $g(t)$ generira linearen cikličen kod dolžine $n=q-1$ nad $GF(q)$ dimenzije $n-p$.

Naj bo $n=2^r-1, \delta\in\{2,\ldots,n\}$, $\beta$ primitiven element $GF(2^r)$. 

\textbf{Reed-Solomonov kod} $RS(n,k)$ je cikličen linearen kod dolžine $n$ in dimenzije $k=n-\delta+1$ nad $GF(2^r)$,
generiran s polinomom $g(t) = (t-\beta)(t-\beta^2)\cdots(t-\beta^{\delta-1})$\\

Naj bo $\mathcal{C}$ Reed-Solomonov kod dolžine $n=2^r-1$ in dimenzije $k$.
Potem je $d(\mathcal{C})=n-k+1$.

\textit{To pomeni, da Reed-Solomonov kod doseže Singletonovo mejo -- popravi največje število napak
glede na št. simbolov, ki jih dodamo sporočilu.}
\end{multicols}
\end{document}
